\documentclass[a4paper,12pt]{article}
\usepackage{czech}
\usepackage[utf8]{inputenc}
\usepackage{a4wide}
\usepackage[dvipdfm]{graphicx}
\usepackage{graphics}
\usepackage{indentfirst}
\usepackage{fancyhdr}
\usepackage{setspace}
\usepackage{amsmath}
\usepackage{amssymb}
\usepackage{epsfig}

%%\usepackage{nopageno}
%%\usepackage{txfonts}
\usepackage[usenames]{color}

\begin{document}
\section{Experiment s rotujícím polarizátorem}
Dříve jsme se zabývali teoretickým popisem magnetooptického experimentu, který využívá závislost intenzity na úhlu, který svírá polarizátor se svislou osou. Nyní se budeme zabývat konkrétní realizací.
\subsection{Světelný zdroj}
Jako zdroj světla používáme heliovou a deuteriovou trubici. Obě najedou pokrývají spektrální rozsah od 250 nm do 1000 nm. Náš zdroj umožňuje použití obou trubic zároveň, avšah heliová trubice vytváří světlo o přibližně pětkrát větší intenzitě. Pokud chceme provést měření celého spektra najednou, musíme odfiltrovat oblast okolo 800 nm, jinak by došlo k přesycení spektrometru dříve, než bychom dosáhli dostatečného signálu ve zbytku spektra.

\subsection{Polarizátory}
V experimentu používáme standartní (DOPLN PRVEK) polarizítoru, které jsou umístěny do držáků s krokovými motorky.  Ty umožňují nastavení úhlu s přesností až $10^{-3}\ ^o$. Jejich ovládání je zprostředkováno kontrolní jednotkou DC 500 od společnosti Owis. Ta umožňuje jejich manuální ovládání i kontrolu přes rozhraní GPIB.

\subsection{Magnet}
Máme hezký chalzený magnet s držákem na vzorek.

\subsection{Spektrometr}
K analýze světla používáme CCD spektrometr USB2000+ od Ocean Optics. Tento spektrometr umožňuje měření celého spektra najednou. Jeho schéma naleznete na obrázku (\ref{USB2000+ schema}). CCD čip uvnitř měří na 2048 pixelech, což v kombinaci s přesností kalibrační křivky umožňuje určení vlnové délky s přesností x.xx nm. Komunikace se spektrometrem je zprostředkována přes rozhraní USB protokolem VISA. Spektrometr podporuje pouze základní bitovou komunikaci, ale ta postačuje k nastavení potřebných atributů, jako je integrační doba.

\subsection{Ovládací program}
Samotné měření se dá rozdělit do dvou částí. První je nastavení aparatury a druhá samotné měření. Toto rozdělení respektuje i ovládací prgram.
\subsubsection{Nastavení aparatury}
Nejprve je potřeba nastavit první polarizátor tak abychom získali p-polarizované světlo. Proto za něj umístíme sklíčko tak jak je ukázáno na obrázku (\ref{TPE install}). Díky odrazu pod Brewsterovým úhlem stačí hledat minimum signálu. Pro nastavení sis tačí vybrat jednu vlnovou dálku, která je v našem případě reprezentována pixelem číslo 808, kde je přibližně maximum heliové trubice. Nyní následuje krátká kapitolka popisující proceduru hledání minima.

\subsubsection{Minseeker}
Celá procedura je založena na znalosti toho, že námi vyšetřovaná závislost má pouze jedno minimum a je konvexní na okolí hrubého kroku, který bude použit dále.

Nejprve provedeme hrubé proměření celé závislosti. V průběhu proměření si držíme pozici nejnižší hodnoty a nižší z okolních hodnot. Jakmile známe hrubý interval, na kterém se minimum nachází nachází, můžeme přistoupit k rekurentnímu dohledání. Vzhledem k nutnosti malé přesnosti a mělkosti minima jsem zvolil ne příliš efektvní ale dostatečně rychlý algoritmus. Tento algoritmus postupně půlí interval a volí směr posunu v závislosti na porovnání hodnot na krajích intervalu, dokud nedosáhnemé kýžené přesnosti.

%\hline

Následně nastavíme aparaturu do uspořádání pro měření. Dále potřebujeme nastavit fruhý polarizátor. Mohli bychom opakovat postup hledání minima u prvního z polarizátorů, ale vzhledem k lepší ostrosti maxima, používáme anologický alogritmus pro hledání maxima a následně polarizátor otočíme o 90 stupňů, čímž docílíme výrazně lepšího zkřížení.

\subsection{Měření}
Na začátku měření program vyžaduje pouze zadání rozsahu a kroku, ve kterém má měřit a četnost měření spekter. Zbytek již obstará sám. Nejprve proběhne měření při první magnetizaci, následně při magnetizaci opačné. Obě části jsou dále vyhodnocovány vlášť a až výsledky se zprůměrují. Po měření se nejprve data přeuspořádají, abychom získali závislost intenzity na úhlu polarizátoru pro jednotlivé vlnové délky. Z každé z nich nalezneme maximum ze kterého určíme normovací konstantu. Dále metodou njmenších čtverců stanovíme buď Kerrovu rotaci, nebo elipticitu v závislosti na přítomnosti fázpvých destiček, a to porovnáním s teoretickou závislostí uvedenou výše. Nakonec jsou výsledná data uloženy do externího souboru.
\end{document}
