\chapter{EM vlny v anizotropním prostředí}
\section{Vlnová rovnice v anizotropním prostředí}
Vlnová rovnice pro světlo v anizotropním prostředí, tedy homogení nevodivé bez nábojů a proudů, se dá snadno dovodit z Maxwellových rovnic
\begin{eqnarray}
\nabla \times\vec{H} - \frac{\partial\vec{D}}{\partial t} = 0 , \label{Max1} \\
\nabla \cdot \vec{D} = 0, \label{Max2} \\
\nabla \times \vec{E} + \frac{\partial \vec{B}}{\partial t} = 0, \label{Max3}\\
\nabla\cdot\vec{B}=0 \label{Max4}
\end{eqnarray}
Na optických frekvencích můžeme počítat, že $\mu_r\approx1$ a tedy platí brát $\mu=\mu_0$. Prostředí je tedy charakterizováno tensorem permitivity $\varepsilon$. Tento tenzor má obecně tvar
\begin{eqnarray}
\varepsilon=
\begin{bmatrix}
\varepsilon_{xx} & \varepsilon_{xy} & \varepsilon_{xz} \\
\varepsilon_{yz}& \varepsilon_{yy}& \varepsilon_{yz} \\
\varepsilon_{zx}& \varepsilon_{zy}& \varepsilon_{zz}
\end{bmatrix}
\end{eqnarray}
Dále vhodné zavést tzv. redukovaný vlnový vektor, který zančíme $\vec{N}$ a je definován
\begin{eqnarray}
\vec{N}=\frac{c}{\omega}\vec{k} = (N_x\vec{i_x}+N_y\vec{i_y}+N_z\vec{i_z})
\end{eqnarray}
Standartní řešení ve tvaru rovinné vlny
\begin{eqnarray}
\vec{E} = \vec{E_0}e^{[i(\omega t-\vec{k}\cdot\vec{r})]}, \\
\vec{B} = \vec{B_0}e^{[i(\omega t-\vec{k}\cdot\vec{r})]}
\end{eqnarray}
dává po dosazení do rovnic (\ref{Max1}) a (\ref{Max2})
\begin{eqnarray}
\vec{k}\times(\vec{k}\times\vec{E}) + \frac{\omega^2}{c^2}\varepsilon\vec{E}=0
\end{eqnarray}
Tuto vektorovou rovnici můžeme rozepsat pro složky vektorů za pomoci Levi-Civitova symbolu $\epsilon$ do tvaru
\begin{eqnarray}
\epsilon_{ijk}k_j\epsilon_{klm}k_lE_m+\frac{\omega^2}{c^2}\varepsilon_{ij}E_j =0
\end{eqnarray}
Postupnou úpravou, která je podrobněji popsána v \ref{Nyvlt} a volbou soustavy souřadné, kde BÚNO $N_x=0$ získáme maticovou rovnici
\begin{eqnarray}
\begin{bmatrix}
\varepsilon_{xx}-N_y^2-N_z^2& \varepsilon_{xy}& \varepsilon_{xz} \\
\varepsilon_{yx}&   \varepsilon_{yy}-N_z^2& \varepsilon_{yz}+N_yN_z\\
\varepsilon_{zx}&   \varepsilon_{zy}+N_yN_z& \varepsilon_{zz}-N_y^2
\end{bmatrix}
\begin{bmatrix}
E_x\\ E_y\\ E_z
\end{bmatrix} = 0
\label{Matic1}
\end{eqnarray}
Pro získání jendoznačných řešení musíme fixovat další parametr, z toho důvodu budeme dále předpokládat znalost komponenty $N_y$, kterou určuje úhel dopadu. %TODO ověřit

Řešení rovnice (\ref{Matic1}) nebude triviální za předpokladu, že determinant první matice bude nulový. Tak získáme charakteristickou rovnici soustavy pro vlastní honodty $N_z$
\begin{eqnarray}
N_z^4\varepsilon_{zz}+N_z^3[N_y(\varepsilon_{yz}+\varepsilon_{yz})]\\
-N_z^2[\varepsilon_{zz}(\varepsilon_{zz}-N_y^2)+\varepsilon_{zz}(\varepsilon_{xx}-N_y^2)-\varepsilon_{xz}\varepsilon_{zx}-\varepsilon_{yz}\varepsilon_{zy}]\\
-N_z[(\varepsilon_{xx}-N_y^2)(\varepsilon_{yz}+\varepsilon_{zy})-\varepsilon_{xy}\varepsilon_{zx}-\varepsilon_{yx}\varepsilon_{xz}]N_y \\
+\varepsilon_{yy}[(\varepsilon_{xx}-N_y^2)(\varepsilon_{zz}-N_y^2)-\varepsilon_{xz}\varepsilon_{zx}]\\
-\varepsilon_{xy}\varepsilon_{yx}(\varepsilon_{zz}-N_y^2)-\varepsilon_{yz}\varepsilon_{zy}(\varepsilon_{xx}-N_y^2)1\varepsilon_{xy}\varepsilon_{zx}\varepsilon_{yz}+\varepsilon_{yx}\varepsilon_{xz}\varepsilon_{zy}=0
\end{eqnarray}
Řešení jsou tedy čtyři hodnoty $N_z$, které popisují šíření čtyř vlastních polarizací $\vec{e}_j$, kde
\begin{eqnarray}
\vec{e}_j=
\begin{bmatrix}
-\varepsilon_{xy}(\varepsilon_{zz}-N_y^2)+\varepsilon_{xz}(\varepsilon_{zy}+N_yN_{zj} \\
(\varepsilon_{zz}-N_y^2)(\varepsilon_{xx}-N_y^2-N^2_{zj})-\varepsilon_{xz}\varepsilon_{zx} \\
-(\varepsilon_{xx}-N_y^2-N^2_{zj})(\varepsilon_{zy}+N_yN_{zj})+\varepsilon_{zx}\varepsilon_{xy}
\end{bmatrix}
\end{eqnarray}
Tato řešení se při průchodu prostředím nemění, proto jsou vhodnou volbou pro bázi. Libovolné pole pak můžeme zapsat ve tvaru
\begin{eqnarray}
\vec{E}=\sum_{j=1}^2E_{0j}\vec{e_j}e^{i\omega t-i\frac{\omega}{c}\vec{N_j}\cdot\vec{r}}
\label{Rozpis E}
\end{eqnarray}

\section{Šíření světla podél vektoru magnetizace}
V případě, kdy se světlo šíří ve směru směru vektoru magnetizace víme díky symetrii problému, že tezor permitivity má výrazně jednodušší tvar
\begin{eqnarray}
\varepsilon=\begin{bmatrix}\varepsilon_{xx}&  -i\varepsilon_{xy}& 0 \\ i\varepsilon{xy}& \varepsilon_{xx}&  0 \\ 0&0& \varepsilon_{zz}\end{bmatrix}
\label{epsilon polar}
\end{eqnarray}
Díky tomu se rovnice (\ref{Matic1}) výrazně zjednodušší. Ve zkratce, pokud zvolíme $N_y=0$, což odpovídá kolmému dopadu, pak se charakteristická ropvnice redukuje na
\begin{eqnarray}
N_z^4-2\varepsilon_1N_z^2+\varepsilon_1^2-\varepsilon_2^2=0,
\end{eqnarray}
což vede na řešení
\begin{eqnarray}
N_z^2=\varepsilon_1 \pm \varepsilon_2.
\end{eqnarray}
Z čehož získáme řádné módy šíření
\begin{eqnarray}
N_\pm=\sqrt{\varepsilon_1\pm\varepsilon_2}
\end{eqnarray}


\section{Magnetické multivrstvy}
Nyní se budeme zabývat situací, kdy máme několik tenkých anizotropních planpalarelních vrstev na sobě. Formalismus popisující tuto problematiku zavedl Yeh. %TODO přidat referenci

Jak bylo zmíněno, předpokládáme materiál tvořený $m$ vrstvami. Rozhraní mezi vrstvami jsou kolmá na osu z. N-tá vrstva je charakterizovaná tenzorem permitivity $\varepsilon^{(n)}$ a tloušťkou $t_n$. Vlnový vektor $\vec{k}_0$ popisující dopadající vlnu svírá s osou z úhel $\varphi$. Elektrické pole v $n$-té vrstvě pak můžeme, jak bylo popsáno výše, rozložit do řádných módů. Tak získáme pro výsledné pole v každé vrstvě výraz
\begin{eqnarray}
\vec{E}^{(n)}=\sum^4_{j=1}E_{0j}^{(n)}(z_n)\vec{e}_j^{(n)}\mbox{exp}\left\{i\omega t-i\frac{\omega}{c}[N_yy+N_{zj}^{(n)}(z-z_n)]\right\},
\end{eqnarray}
kde $z_n$ značí z-ovou souřadnici rozhraní $n$-té a $(n+1)$-ní vrstvy a $N_{zj}$ komponenty redukovaného vlnového vektoru.

Dále bez odvození uvádíme okrajové podmínky na rozhraní $n$-té a $(n-1$)-ní vrstvy. Jedná se o soustavu čtyř rovnic pro $E_x$, $E_y$, $B_y$ a $B_x$ %TODO ověřit typografii
\begin{eqnarray}
\sum^4_{j=1}E_{0j}^{(n-1)}(z_{n-1})\vec{e}_j^{(n-1)}\cdot\vec{i}_x &=& \sum^4_{j=1}E_{0j}^{(n)}(z_n)\vec{e}_j^{(n)}\cdot\vec{i}_x \mbox{exp}\left(i\frac{\omega}{c}N_{zj}^{(n)}t_n\right), \\
\sum^4_{j=1}E_{0j}^{(n-1)}(z_{n-1})\vec{b}_j^{(n-1)}\cdot\vec{i}_y &=& \sum^4_{j=1}E_{0j}^{(n)}(z_n)\vec{b}_j^{(n)}\cdot\vec{i}_y \mbox{exp}\left(i\frac{\omega}{c}N_{zj}^{(n)}t_n\right), \\
\sum^4_{j=1}E_{0j}^{(n-1)}(z_{n-1})\vec{e}_j^{(n-1)}\cdot\vec{i}_y &=& \sum^4_{j=1}E_{0j}^{(n)}(z_n)\vec{e}_j^{(n)}\cdot\vec{i}_y \mbox{exp}\left(i\frac{\omega}{c}N_{zj}^{(n)}t_n\right), \\
\sum^4_{j=1}E_{0j}^{(n-1)}(z_{n-1})\vec{b}_j^{(n-1)}\cdot\vec{i}_x &=& \sum^4_{j=1}E_{0j}^{(n)}(z_n)\vec{b}_j^{(n)}\cdot\vec{i}_x \mbox{exp}\left(i\frac{\omega}{c}N_{zj}^{(n)}t_n\right).
\end{eqnarray}
Tato soustava popisuje lineární transformaci amplitud příslušných modů. Velmi výhodné je její přepsání do maticové rovnice
\begin{eqnarray}
\mathbb{D}^{(n-1)}\vec{E}_0^{(n-1)}(z_{n-1})=\mathbb{D}^{(n)}\mathbb{P}^{(n)}\vec{E}^{(n)}_0(z_n),
\end{eqnarray}
kde čtvrtá komponenta vektoru $\vec{E_0^{(n)}}$ je koeficient $E^{(n)}_{0j}(z_n)$. Prvky propagační matice $\mathbb{P}$ jsou dány
\begin{eqnarray}
\mathbb{P}_{ij}^{(n)}=\delta_{ij} \mbox{exp}\left(i\frac{\omega}{c}N_{zj}^{(n)}t_n\right).
\end{eqnarray}
Řádky dynamické matice $\mathbb{D}$ jsou pak dány komponentami příslušných polarizací
\begin{eqnarray}
\mathbb{D}_{1j}^{(n)}=\vec{e}_j^{(n)}\cdot\vec{i}_x, \\
\mathbb{D}_{2j}^{(n)}=\vec{b}_j^{(n)}\cdot\vec{i}_y, \\
\mathbb{D}_{3j}^{(n)}=\vec{e}_j^{(n)}\cdot\vec{i}_y, \\
\mathbb{D}_{4j}^{(n)}=\vec{b}_j^{(n)}\cdot\vec{i}_x.
\end{eqnarray}
Abychom se nemuseli zabývat obencným řešením těchto rovnic, využijeme toho, že při polární magnetizaci má tenzor permitivity tvar (\ref{epsilon polar}). To vede na zjednodušené rovnice
\begin{eqnarray}
\mathbb{D}_{1j}^{(n)}&=& -\varepsilon_{xy}(\varepsilon_{xx}-N_y^2)+\varepsilon_{xx}N_yN_z, \\
\mathbb{D}_{2j}^{(n)}&=&N_{zj}[-\varepsilon_{xy}^{(n)}(\varepsilon_{zz}^{(n)}-N_y^2)], \\
\mathbb{D}_{3j}^{(n)}&=&(\varepsilon_{zz}^{(n)}-N^2_y(\varepsilon_{xx}^{(n)}-N_y^2-N_{zj}^{(n)2}), \\
\mathbb{D}_{4j}^{(n)}&=&-(\varepsilon_{xx}^{(n)}-N_y^2-N_{zj}^{(n)2})N_{zj}^{(n)}\varepsilon_{zz}^{(n)})
\end{eqnarray}
díky kterým sjme schopni určit celou dynamickou matici.
Pro $m$ vrstev pak získáme výsledný vztah pouhým násobením matic
\begin{eqnarray}
\vec{E}_0^{(0)}(z_0) = [\mathbb{D}^{(0)}]^{-1}\mathbb{D}^{(1)}\mathbb{P}^{(1)}[\mathbb{D}^{(1)}]^{-1}\dots \mathbb{D}^{(m)}\mathbb{P}^{(m)}[\mathbb{D}^{(m)}]^{-1}\mathbb{D}^{(m+1)}\vec{E}_0^{(m+1)}(z_m)\\
=\mathbb{M}\vec{E}_0^{(m+1)}(z_m)
\end{eqnarray}
Z matice $\mathbb{M}$ můžeme následně vypočítat reflexní a trasmisní koeficienty, jako poměr apmlitudy dopadajícího a odraženého, či prošlého pole, s uvážením, že z $E^{(n)}$ nic nedopadá. Ve zkratce získáme vztahy %TODO ověřit
\begin{eqnarray}
r_{12}=\frac{M_{21}M_{33} - M_{23}M_{31}}{M_{11}M_{33}-M_{13}M_{31}}, \\
r_{14}=\frac{M_{41}M_{33}-M_{43}M_{31}}{M_{11}M_{33}-M_{13}M_{31}}, \\
r_{31}=\frac{M_{11}M_{43}-M_{41}M_{13}}{M_{11}M_{33}-M_{13}M_{31}}, \\
r_{32}=\frac{M_{11}M_{23}-M_{21}M_{13}}{M_{11}M_{33}-M_{13}M_{31}}.
\end{eqnarray}
Přičemž platí vztah vázající tyto koeficienty s Jonesovou reflexní maticí
\begin{eqnarray}
\begin{bmatrix}
r_{ss} & r_{sp} \\ r_{ps} & r_{pp} 
\end{bmatrix}
= 
\begin{bmatrix} 
r_{12} &r_{32} \\ -r_{14} & -r_{34} 
\end{bmatrix}
\end{eqnarray}
Analogické vztahy platí i pro transmisní koeficienty. Z obou pak můžeme dopočítat příslušné Kerrovy či Faradayovy koeficienty.
