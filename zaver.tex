\chapter*{Závěr}
\addcontentsline{toc}{chapter}{Závěr}

Náplní této práce bylo zkoumání metod měření magnetooptických jevů. Kvůli tomu, byla nejprve rozebrán popis polarizace světla 
a její vývoj v prostoru za pomoci Jonesova formalismus, v rámci čehož byli zavedeny i magnetooptické veličiny. Tento formalizmus 
byl posléze využit pro popis teoretického podkladu experimentálních metod, kterými jsme se zabývali.

Následně byla vyložena teorie týkající se anizotropních látek, jejíž závěrem bylo teoretické určení magnetooptických veličin pro jednoduchou 
vrstu. Toho bylo využito pro výpočet spekter tenkých vrstev La$_{2/3}$Sr$_{1/3}$MnO$_3$, která byla srovnána naměřenými hodnotami. 
Pro zbytek vzorků sjem neměli potřebné materiálové konstanty, abychom mohli vypočítat teoretické spektrum.

Dále jsme se zabývali samotnými experimentálními metodami měření Kerrova jevu. K tomuto tématu byly uvedeny dvě různé metody. Ukaždé znich 
jsme rozebrali teoretický model, na jehož základě jsme postavili celou aparaturu pro metodu zkřížených polarizátorů. 
Znalosti modulační metody byli využity k modernizaci ovládacího programu této metody.

Na závěr byli změřeny konkrétní vzorky ze zástupců tří odlišných skupin materiálů a na nich porovnány účinnosti obou metod. Jako lepší metoda
se prokázala metoda s téměř zkříženými polarizátory. Mezi její hlavní přednosti patří výrazně kratší čas měření, lepší rozlišovací schopnost, 
konstrukční jednoduchost, variabilita a spolehlivost. Za zmínku také stojí výrazně nížší pořizovací náklady, a to především kvůli menšímu počtu 
zařízení a slabšímu zdorji světla.
