\chapter*{Závěr}
\addcontentsline{toc}{chapter}{Závěr}

Náplní této práce bylo zkoumání metod měření magnetooptických jevů. Kvůli tomu, byl nejprve rozebrán popis polarizace světla 
a její vývoj v prostoru za pomoci Jonesova formalismu, v rámci čehož byly zavedeny i magnetooptické veličiny. Tento formalizmus 
byl posléze využit pro popis teoretického podkladu experimentálních metod, kterými jsme se zabývaly.

Následně byla vyložena teorie týkající se anizotropních látek, jejíž závěrem bylo teoretické určení magnetooptických veličin pro jednoduchou 
vrstu. Toho bylo využito pro výpočet spekter tenkých vrstev La$_{2/3}$Sr$_{1/3}$MnO$_3$, která byla srovnána s naměřenými hodnotami. 
Pro zbytek vzorků jsme neměli potřebné materiálové konstanty nutné pro výpočet.

Dále jsme se zabývali samotnými experimentálními metodami měření Kerrova jevu. K tomuto tématu byly uvedeny dvě různé metody. U každé znich 
jsme rozebrali teoretický model a u metody zkřížených polarizátorů jsme na jeho základě jsme postavili celou aparaturu. 
Znalosti modulační metody byly využity k modernizaci ovládacího programu této metody.

Na závěr byly změřeny konkrétní vzorky ze zástupců tří odlišných skupin materiálů a na nich porovnány účinnosti obou metod. Jako lepší metoda
se prokázala metoda se zkříženými polarizátory. Mezi její hlavní přednosti patří výrazně kratší čas měření, lepší rozlišovací schopnost, 
konstrukční jednoduchost, variabilita a spolehlivost. Za zmínku také stojí výrazně nížší pořizovací náklady, a to především kvůli menšímu počtu 
zařízení, optických prvků a slabšímu zdroji světla.
