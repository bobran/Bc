\chapter{Vzorky}

\section{Ultratenké vrstvy La$_{2/3}$Sr$_{1/3}$MnO$_3$}
První vzorek patří do skupiny vytvářené metodou pulsní laserové deposice (dále PLD). Tato metoda umožňuje vytváření tenkých filmů o tloušťkách menších než 50 nm. Ve zkratce metoda funguje na principu kondenzace plasmy odpařované za pomoci krátkých laserových pulsů. V našem případě kondenzace probíhá na substrátu z SrTiO$_3$. Použité vzorky měli tloušťku 23,1 nm (PLD 186) a 80,8 nm (PLD202). 

\section{Kobaltové feritové tenké vrstvy}
Ze zástupců magnetických oxidů jsme vybrali nanovrstvy CoFe$_2$O$_4$. Tento materiál je díky jeho magnetickým vlastnostem často používán v mikrovlnných zařízeních a  pro magnetooptický zípis. Jejich hlavní předností je velká Curiova teplota, vysoká koercitiva a veliká magnetická anizotropie. Naše vzorky byli depozitovány metodou PLD na substrát z amorfního taveného křemene. K ablaci vzorku byl použit Nd:YAG laser. Teplota při deposici vzorků byla 750 (CoF-RT-A750) a 1100 $^\circ$C (CoF-RT-A1100). Tloušťka 110nm.

\section{Heuslerovy slitiny}
Jako poslední typ vzorků jsme použili zástupce z Heuslerovývh slitin. Námi zkoumané vzorky různou směsí koblatu, železa a křemíku. Substrát vzorku byl MgO. Na něm je 5 nm chrómu, 20 nm slitiny a navrchu 2 nm MgO. Tyto slitiny mají vynikající magnetické vlastnosti a velmi vysokou Curiovu teplotu. V současné době je velký zájem o výzkum těchto slitin, protože jejich vlastnosti nejsou příliš známy.  
