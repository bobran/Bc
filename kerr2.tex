\documentclass[a4paper,12pt]{article}
\usepackage{czech}
\usepackage[utf8]{inputenc}
\usepackage{a4wide}
\usepackage[dvipdfm]{graphicx}
\usepackage{graphics}
\usepackage{indentfirst}
\usepackage{fancyhdr}
\usepackage{setspace}
\usepackage{amsmath}
\usepackage{amssymb}
\usepackage{epsfig}

%%\usepackage{nopageno}
%%\usepackage{txfonts}
\usepackage[usenames]{color}

\begin{document}
\section{Zařízení}
\subsection{Monochromátor}
Nová aparatrura obsahuje monochromátor TRIAX 550. Jedná se o mřížkový monochromátor s možností volby z různých mřížek. V našem případě máme na výběr 600, 900 a 1200 vrypů na mm. Parametry těchto mřížek jsou uvedeny v tabulce (\ref{TTriax}). Tento monochromátor lze ovládat za pomoci rozhraní GPIB. 
\begin{table}
$$
\begin{array}{|c|c|c|}
\hline
\mbox{grating} [g/mm]&  \mbox{Disperze} [nm/mm]&    \mbox{Spektrální rozsah} [nm] \\ \hline
600&    2.83&   0 - 3000 \\ \hline
900&    1.84&   0 - 2000 \\ \hline
1200&   1.34&   0 - 1500 \\ \hline
\end{array}
$$
\caption{Parametry mřížek monochromátoru}
\label{TTriax}
\end{table}

\subsection{Multimeter}
K Určení velikosti signálu na fotonásobiči používáme mulimetr Keithley 2001. V našem případě v měříme napětí na rozsahu 1 V. Možnost automatického rozsahu nepoužíváme, protože při přepnutí rozsahu dochází ke skokům napětí. Komunikace je opět zprostředkována rozhraním GPIB.

%TODO Samý kidy, vymysli co napsat!!!
\subsection{Zroj}
Jako zroj elektrického proudu pro magnet používáme zdroj stejnosměrbého proudu Kepco BOP. Tento zdroj


\section{Ovládací program Kerr2}
Dále se budu zabývat ovladacím programem pro druhý z experimentů uvedených výše. 

\subsection{Nastavení experimentu}
Tento program má hned několik funkcí, které usnadňují nastavení experimetnu. První z nich umožňuje manuální nastavení proudu magnetem. Uživatel zadá požadovaný proud a program pomalu zvyšuje proud, dokud nedosáhne požadované hodnoty. Dálší mód nastaví monochromátor na požadovanou vlnovou délku a otevře štěrbiny. Toho se používá především pro nastavení prvků v experimentu.

\subsection{Měření spektra}
Program umožňuje proměření spektra ve zvoleném rozahu s libovolným krokem. Dále umožňuje nastavení tolerance chyby měření, čekací doby po změňe magnetizace, počet měření jednotlivé vlnové délky a kalibračních koeficientů pro výpočet energie signálu. Po zahájení měření program nejprce nastaví na monochromátoru měřenou vlnovou délku a zapne proud do magnetu. Proud je přidáván postupně kvůli možnému zkratu na zroji při rychlém přepólování. Každé spektrum se měří opakovaně dle zadání uživatele, přičemž v celém prlběhu měření je kontrolováno, zda nebyl překročen rozsah. V takovém případě se měření pozastaví, aby umožnilo manuální otočení polarizátoru a měření pokračuje znovu od poslední vlnové délky. Měření opět porbíhá i pro opačnou magnetizaci, přičemž program umožňuje zadání počtu otáček potřebných pro navrácení do rozsahu po změně polarizace. Nakonec je pro danou vlnovou délku provedeno třetí měření s původní magnetizací a je zkontrolována odchylka od prvního měření. V případě příliš velké odchylky se měření opakuje. V průběhu celého měření je vykreslován graf, ze kterého je možné již při měření odhalit případné nespojitosti. Po skončení měření jsou data uložena do expterního souboru spolu se všemi parametry měření.

\subsection{Hysterzní smyčky}
Program dále umožňuje měření hysterzních smyček a to dvěma způsoby. První postupně proměří při zadané vlnové délce různé hodnoty proudu, přičemž postupuje od zadané hodnoty I do -I s krokem, který je rovněž zadán. Následně se stejným krokem vrátí do hodnoty I. V průběhu měření je opět kreslen graf.

Druhá metoda nese anglický název four loop. Spočívá v postupném proměření hodnot vzdálených o $\Delta$I od hodnoty proudu I resp -I, přičemž tato vzálenost roste se zadaným krokem. Prlběh proudu je znázorněn na obrázku (\ref{4-loop}). Podstatné je, že vždy dojde do hodnoty I resp. -I.

\end{document}
